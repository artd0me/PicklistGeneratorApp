% Options for packages loaded elsewhere
\PassOptionsToPackage{unicode}{hyperref}
\PassOptionsToPackage{hyphens}{url}
%
\documentclass[
]{article}
\usepackage{amsmath,amssymb}
\usepackage{lmodern}
\usepackage{iftex}
\ifPDFTeX
  \usepackage[T1]{fontenc}
  \usepackage[utf8]{inputenc}
  \usepackage{textcomp} % provide euro and other symbols
\else % if luatex or xetex
  \usepackage{unicode-math}
  \defaultfontfeatures{Scale=MatchLowercase}
  \defaultfontfeatures[\rmfamily]{Ligatures=TeX,Scale=1}
\fi
% Use upquote if available, for straight quotes in verbatim environments
\IfFileExists{upquote.sty}{\usepackage{upquote}}{}
\IfFileExists{microtype.sty}{% use microtype if available
  \usepackage[]{microtype}
  \UseMicrotypeSet[protrusion]{basicmath} % disable protrusion for tt fonts
}{}
\makeatletter
\@ifundefined{KOMAClassName}{% if non-KOMA class
  \IfFileExists{parskip.sty}{%
    \usepackage{parskip}
  }{% else
    \setlength{\parindent}{0pt}
    \setlength{\parskip}{6pt plus 2pt minus 1pt}}
}{% if KOMA class
  \KOMAoptions{parskip=half}}
\makeatother
\usepackage{xcolor}
\usepackage[margin=1in]{geometry}
\usepackage{color}
\usepackage{fancyvrb}
\newcommand{\VerbBar}{|}
\newcommand{\VERB}{\Verb[commandchars=\\\{\}]}
\DefineVerbatimEnvironment{Highlighting}{Verbatim}{commandchars=\\\{\}}
% Add ',fontsize=\small' for more characters per line
\usepackage{framed}
\definecolor{shadecolor}{RGB}{248,248,248}
\newenvironment{Shaded}{\begin{snugshade}}{\end{snugshade}}
\newcommand{\AlertTok}[1]{\textcolor[rgb]{0.94,0.16,0.16}{#1}}
\newcommand{\AnnotationTok}[1]{\textcolor[rgb]{0.56,0.35,0.01}{\textbf{\textit{#1}}}}
\newcommand{\AttributeTok}[1]{\textcolor[rgb]{0.77,0.63,0.00}{#1}}
\newcommand{\BaseNTok}[1]{\textcolor[rgb]{0.00,0.00,0.81}{#1}}
\newcommand{\BuiltInTok}[1]{#1}
\newcommand{\CharTok}[1]{\textcolor[rgb]{0.31,0.60,0.02}{#1}}
\newcommand{\CommentTok}[1]{\textcolor[rgb]{0.56,0.35,0.01}{\textit{#1}}}
\newcommand{\CommentVarTok}[1]{\textcolor[rgb]{0.56,0.35,0.01}{\textbf{\textit{#1}}}}
\newcommand{\ConstantTok}[1]{\textcolor[rgb]{0.00,0.00,0.00}{#1}}
\newcommand{\ControlFlowTok}[1]{\textcolor[rgb]{0.13,0.29,0.53}{\textbf{#1}}}
\newcommand{\DataTypeTok}[1]{\textcolor[rgb]{0.13,0.29,0.53}{#1}}
\newcommand{\DecValTok}[1]{\textcolor[rgb]{0.00,0.00,0.81}{#1}}
\newcommand{\DocumentationTok}[1]{\textcolor[rgb]{0.56,0.35,0.01}{\textbf{\textit{#1}}}}
\newcommand{\ErrorTok}[1]{\textcolor[rgb]{0.64,0.00,0.00}{\textbf{#1}}}
\newcommand{\ExtensionTok}[1]{#1}
\newcommand{\FloatTok}[1]{\textcolor[rgb]{0.00,0.00,0.81}{#1}}
\newcommand{\FunctionTok}[1]{\textcolor[rgb]{0.00,0.00,0.00}{#1}}
\newcommand{\ImportTok}[1]{#1}
\newcommand{\InformationTok}[1]{\textcolor[rgb]{0.56,0.35,0.01}{\textbf{\textit{#1}}}}
\newcommand{\KeywordTok}[1]{\textcolor[rgb]{0.13,0.29,0.53}{\textbf{#1}}}
\newcommand{\NormalTok}[1]{#1}
\newcommand{\OperatorTok}[1]{\textcolor[rgb]{0.81,0.36,0.00}{\textbf{#1}}}
\newcommand{\OtherTok}[1]{\textcolor[rgb]{0.56,0.35,0.01}{#1}}
\newcommand{\PreprocessorTok}[1]{\textcolor[rgb]{0.56,0.35,0.01}{\textit{#1}}}
\newcommand{\RegionMarkerTok}[1]{#1}
\newcommand{\SpecialCharTok}[1]{\textcolor[rgb]{0.00,0.00,0.00}{#1}}
\newcommand{\SpecialStringTok}[1]{\textcolor[rgb]{0.31,0.60,0.02}{#1}}
\newcommand{\StringTok}[1]{\textcolor[rgb]{0.31,0.60,0.02}{#1}}
\newcommand{\VariableTok}[1]{\textcolor[rgb]{0.00,0.00,0.00}{#1}}
\newcommand{\VerbatimStringTok}[1]{\textcolor[rgb]{0.31,0.60,0.02}{#1}}
\newcommand{\WarningTok}[1]{\textcolor[rgb]{0.56,0.35,0.01}{\textbf{\textit{#1}}}}
\usepackage{graphicx}
\makeatletter
\def\maxwidth{\ifdim\Gin@nat@width>\linewidth\linewidth\else\Gin@nat@width\fi}
\def\maxheight{\ifdim\Gin@nat@height>\textheight\textheight\else\Gin@nat@height\fi}
\makeatother
% Scale images if necessary, so that they will not overflow the page
% margins by default, and it is still possible to overwrite the defaults
% using explicit options in \includegraphics[width, height, ...]{}
\setkeys{Gin}{width=\maxwidth,height=\maxheight,keepaspectratio}
% Set default figure placement to htbp
\makeatletter
\def\fps@figure{htbp}
\makeatother
\setlength{\emergencystretch}{3em} % prevent overfull lines
\providecommand{\tightlist}{%
  \setlength{\itemsep}{0pt}\setlength{\parskip}{0pt}}
\setcounter{secnumdepth}{-\maxdimen} % remove section numbering
\ifLuaTeX
  \usepackage{selnolig}  % disable illegal ligatures
\fi
\IfFileExists{bookmark.sty}{\usepackage{bookmark}}{\usepackage{hyperref}}
\IfFileExists{xurl.sty}{\usepackage{xurl}}{} % add URL line breaks if available
\urlstyle{same} % disable monospaced font for URLs
\hypersetup{
  hidelinks,
  pdfcreator={LaTeX via pandoc}}

\author{}
\date{\vspace{-2.5em}}

\begin{document}

\hypertarget{picklistgeneratorapp}{%
\section{PicklistGeneratorApp}\label{picklistgeneratorapp}}

This app was designed to create picklists compatible with the Echo
Liquid handling systems.\\
The current purpose is to shoot fluorescent antibody panels to stain
cells for flow cytometry.\\
Using this app reduces the time for preparing the picklist while
including controls such as FMOs, isotype controls and single stains for
compensation.

\hypertarget{getting-started}{%
\subsection{Getting started}\label{getting-started}}

\hypertarget{download-the-app}{%
\subsection{Download the app}\label{download-the-app}}

To download the app, click on the green ``\textless\textgreater{} Code''
button and choose ``Download ZIP''. Save the whole folder in a directory
of your choice.

IF YOU ARE WORKING ON MAC: Open the ``RunAppMac.sh'' file in an editor
(e.g.~TextEdit) and replace the pathname with the new path in which you
saved the ``gui.py'' file, e.g:

\begin{Shaded}
\begin{Highlighting}[]
\NormalTok{python /your/path/to/gui.py}
\end{Highlighting}
\end{Shaded}

\hypertarget{installation-of-python-on-windows}{%
\subsubsection{Installation of Python on
Windows}\label{installation-of-python-on-windows}}

To use this app, install Python on your computer.\\
To verify that Python is working, use the Windows Command Prompt:\\
1. Press Windows key + R.\\
2. Type ``cmd'' and press enter.

Once you are in the Command Prompt, type

\begin{Shaded}
\begin{Highlighting}[]
\NormalTok{python {-}{-}version}
\end{Highlighting}
\end{Shaded}

If the output does not show the current Python version and you're
encountering the ``python is not recognized as an internal or external
command'' error in the Windows Command Prompt, it usually means that the
Python executable is not properly added to your system's PATH
environment variable during the installation. Here's how you can fix
this issue:

\begin{enumerate}
\def\labelenumi{\arabic{enumi}.}
\tightlist
\item
  Check Python Installation Path: Verify the path where Python is
  installed on your system.\\
\item
  Add Python to PATH: You need to add the Python installation directory
  to your system's PATH variable. Follow these steps:\\
\item
  Right-click on the ``This PC'' or ``My Computer'' icon on your desktop
  or in the File Explorer and select ``Properties.'' Click on ``Advanced
  system settings'' (you might need administrator privileges).\\
\item
  In the System Properties window, click the ``Environment Variables''
  button. In the Environment Variables window, under the ``System
  variables'' section, find and select the ``Path'' variable, then click
  the ``Edit'' button.\\
\item
  Click the ``New'' button and add the path to your Python installation
  directory.\\
\item
  Click ``OK'' to close all the windows.\\
\item
  Restart Command Prompt: After modifying the PATH environment variable,
  close any open Command Prompt windows and reopen a new one. Try
  running
\end{enumerate}

\begin{Shaded}
\begin{Highlighting}[]
\NormalTok{python {-}{-}version}
\end{Highlighting}
\end{Shaded}

again to see if the issue is resolved.\\

\begin{enumerate}
\def\labelenumi{\arabic{enumi}.}
\setcounter{enumi}{7}
\tightlist
\item
  Restart Your Computer: If the issue persists, try restarting your
  computer after modifying the PATH variable to ensure the changes take
  effect.
\end{enumerate}

After completing these steps, the ``python is not recognized'' error
should be resolved, and you should be able to run Python commands from
the Command Prompt without any issues. If you continue to experience
problems, double-check that you followed the steps correctly and
consider reinstalling Python, making sure to select the option to add
Python to the PATH during installation.

\hypertarget{installation-of-pandas}{%
\subsubsection{Installation of pandas}\label{installation-of-pandas}}

In the Command Prompt, run

\begin{Shaded}
\begin{Highlighting}[]
\NormalTok{pip install pandas}
\end{Highlighting}
\end{Shaded}

to install the pandas package needed for the execution of the App.

After the installation is complete, you can verify that Pandas is
installed by opening a Python interactive shell. Type

\begin{Shaded}
\begin{Highlighting}[]
\NormalTok{python}
\end{Highlighting}
\end{Shaded}

in the Command Prompt and press Enter to open the Python shell, then
type the following:

\begin{Shaded}
\begin{Highlighting}[]
\NormalTok{import pandas as pd}
\NormalTok{print(pd.\_\_version\_\_)}
\end{Highlighting}
\end{Shaded}

This should display the version number of the Pandas library that was
installed.

\hypertarget{preparation-of-the-input-files}{%
\subsubsection{Preparation of the input
files}\label{preparation-of-the-input-files}}

\hypertarget{the-panel-library-reference-file}{%
\paragraph{The Panel library / Reference
file}\label{the-panel-library-reference-file}}

The panel library file serves as a reference that contains the layout of
the source plate and defines the design of each panel and it's
modifications. In the example file ``ExamplePanelLibrary.csv'', we have
9 different columns defining the panel:

\begin{itemize}
\tightlist
\item
  Panel: the name of the panel, e.g.~TCellpanel, Panel1 etc.\\
\item
  Marker: Which molecule does this antibody bind to?\\
\item
  Color: What fluorophor is the antibody conjugated to?\\
\item
  Modification: is the panel complete with all antibodies, or is it an
  FMO panel (FMOCD3 doesn't include CD3), only a single stain or IgG
  control?\\
\item
  Panel\_Modification: merge between Panel and Modification, works as
  the ``key'' word to connect to the destination plate layout
  (AnnotationFile). DO NOT CHANGE THE NAME OF THIS COLUMN!\\
\item
  Source Well: In which well of the source plate is the antibody? DO NOT
  CHANGE THE NAME OF THIS COLUMN!\\
\item
  nl\_per\_20: how much nl of the antibody do you need to transfer per
  20 µl cell suspension. DO NOT CHANGE THE NAME OF THIS COLUMN!\\
\item
  Source Plate Barcode: the name of your source plate! Use the same
  name, if you use only one source plate! If you have the antibodies
  distributed to several source plates, give them different names! DO
  NOT CHANGE THE NAME OF THIS COLUMN!
\end{itemize}

!! Make sure that the .csv-file uses commas to separate columns '' , ''
!!

This file serves as a masterfile, that you can continuously update with
new panels or if your source plate layout changes!

\hypertarget{the-destination-plate-annotation-file}{%
\paragraph{The Destination plate / Annotation
file}\label{the-destination-plate-annotation-file}}

The annotation file serves as the layout of your destination plate,
where you define which panels go into which well.\\
In the example file ``ExampleDestinationPlateAnnotation.csv'', we have 5
different columns:

\begin{itemize}
\tightlist
\item
  Destination Plate Barcode: the name of your destination plate! Use the
  same name, if you only have one destination plate. If you have several
  destination plates to dispense to, give them different names (only if
  different layouts are needed. If you do replicas, you can reuse the
  same annotation file) DO NOT CHANGE THE NAME OF THIS COLUMN!\\
\item
  Destination Well: Well, to which the antibodies are transferred to. DO
  NOT CHANGE THE NAME OF THIS COLUMN!\\
\item
  Panel\_Modification: Describes with which panel and modicifation you
  want to stain this well. Needs to be the exact value as in the panel
  library, as it works as the ``key'' word to connect to the source
  plate layput (panel library). DO NOT CHANGE THE NAME OF THIS COLUMN!\\
\item
  cell\_volume: How big is your suspension volume (in µl)? Needed for
  calculating the final transfer volume! DO NOT CHANGE THE NAME OF THIS
  COLUMN!\\
\item
  sampledescription: your description to identify/annotate the samples!
\end{itemize}

!! Make sure that the .csv-file uses commas to separate columns '' , ''
!!

\hypertarget{running-the-app}{%
\subsubsection{Running the app}\label{running-the-app}}

\hypertarget{starting-it-up-on-windows}{%
\paragraph{Starting it up on Windows}\label{starting-it-up-on-windows}}

Double click the file ``RunAppWindows.py''

\hypertarget{generate-picklists}{%
\paragraph{Generate picklists}\label{generate-picklists}}

Once the window opens, you can browse your annotation and library file,
select a folder to save your picklist, and paste the name of the
experiment!

Click ``Generate Picklist''.

The picklist should be saved in the folder you selected and you can
close the window again!

\hypertarget{the-picklist}{%
\subsubsection{The picklist}\label{the-picklist}}

The picklist name has the following structure:

DATE\_Picklist\_ExperimentName

\begin{itemize}
\tightlist
\item
  DATE: the date when you generated the picklist in a YYYYMMDD format\\
\item
  ExperimentName: The name you have given it in the last step while
  running the app!
\end{itemize}

The picklist contains the following columns:

\begin{itemize}
\tightlist
\item
  Source Plate Barcode: see previously\\
\item
  Source Well: See previously\\
\item
  Destination Plate Barcode: see previously\\
\item
  Destination Well: see previously\\
\item
  Transfer Volume: the final volume the Echo will be shooting from
  source to destination well!
\end{itemize}

The picklist can be loaded into the Echo system, without any further
unticking or ticking of any boxes!

\end{document}
